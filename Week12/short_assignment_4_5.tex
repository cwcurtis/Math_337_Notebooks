 \documentclass[a4paper,11pt]{article}

\usepackage{amsmath}
\usepackage{amssymb}
\usepackage{amsthm}
\usepackage{graphicx}
\usepackage{caption}
\usepackage{subcaption}

\newtheorem{thm}{Theorem}
\newtheorem{dfn}{Definition}
\newtheorem{lem}{Lemma}

\newcommand{\beq}{\begin{equation}}
\newcommand{\eeq}{\end{equation}}

\newcommand{\ba}{\begin{array}}
\newcommand{\ea}{\end{array}}

\newcommand{\bea}{\begin{eqnarray}}
\newcommand{\eea}{\end{eqnarray}}

\newcommand{\bc}{\begin{center}}
\newcommand{\ec}{\end{center}}

\newcommand{\ds}{\displaystyle}

\newcommand{\bt}{\begin{tabular}}
\newcommand{\et}{\end{tabular}}

\newcommand{\bi}{\begin{itemize}}
\newcommand{\ei}{\end{itemize}}

\newcommand{\bd}{\begin{description}}
\newcommand{\ed}{\end{description}}

\newcommand{\bp}{\begin{pmatrix}}
\newcommand{\ep}{\end{pmatrix}}

\newcommand{\p}{\partial}
\newcommand{\sech}{\mbox{sech}}

\newcommand{\cf}{{\it cf.}~}

\newcommand{\ltwo}{L_{2}(\mathbb{R}^{2})}
\newcommand{\smooth}{C^{\infty}_{0}(\mathbb{R}^{2})}

\newcommand{\br}{{\bf r}}
\newcommand{\bk}{{\bf k}}
\newcommand{\bv}{{\bf v}} 
\newcommand{\bu}{{\bf u}}
\newcommand{\bdd}{{\bf d}}
\newcommand{\bpp}{{\bf p}}
\newcommand{\bm}{{\bf m}}
\newcommand{\bn}{{\bf n}}

\newcommand{\pt}{\mathcal{K}}

\newcommand{\gnorm}[1]{\left|\left| #1\right|\right|}
\newcommand{\Lnorm}{L^{2}\left(\mathbb{R}^{2}\right)}
\newcommand{\ipro}[2]{\left<#1,#2 \right>}
\pagestyle{empty}
\begin{document}

\begin{center}
{\bf Gradescope Assignment: Due 4/7/21\\
0 pts for no work\\ 2 pts for attempt\\ 4 pts for full answer}
\end{center}

\begin{enumerate}
\item 4.6.14.  You could use the book's advice, but that would make you a weenie.  A better approach is to use the formulas we derived in the notes (see Week 11, page 2 of the part two notes).  You do not need to rederive the formulas presented there.  

\item For the system,
\[
a\frac{d^{2}}{dt^{2}}x + b \frac{d}{dt}x + c x = \cos(\nu t),
\]
we showed in the underdamped case, where $b^{2}<4ac$, that the particular solution is given by 
\begin{align*}
x_{p}(t) = & \frac{1}{2\omega}\left(\frac{\mu}{\mu^{2}+(\nu-\omega)^{2}}\left(\sin(\nu t) - \sin(\omega t)e^{-\mu t}\right) \right. \\
& + \frac{(\nu-\omega)}{\mu^{2}+(\nu-\omega)^{2}}\left(-\cos(\nu t) + \cos(\omega t)e^{-\mu t}\right) \\
& - \frac{\mu}{\mu^{2}+(\nu+\omega)^{2}}\left(\sin(\nu t) + \sin(\omega t)e^{-\mu t}\right) \\
& \left. - \frac{(\omega+\nu)}{\mu^{2}+(\nu+\omega)^{2}}\left(-\cos(\nu t) + \cos(\omega t)e^{-\mu t}\right) \right)
\end{align*}
where $\mu=b/2a$ and $\omega = \sqrt{4ac-b^{2}}/2a$.  Thus, if $b=0$, $\mu=0$, and we have 
\[
x_{p}(t) = \frac{1}{\nu^{2}-\omega^{2}}\left(\cos(\omega t)-\cos(\nu t)\right) 
\]
See also Equation 19 on page 270 of the book.  Using their suggestion, provide the necessary steps to show Equation 20 on page 270.  Briefly explain how "beats" are generated in this case.  See Figure 4.6.8 for reference.  
\end{enumerate}
\end{document}