 \documentclass[a4paper,11pt]{article}

\usepackage{amsmath}
\usepackage{amssymb}
\usepackage{amsthm}
\usepackage{graphicx}
\usepackage{caption}
\usepackage{subcaption}

\newtheorem{thm}{Theorem}
\newtheorem{dfn}{Definition}
\newtheorem{lem}{Lemma}

\newcommand{\beq}{\begin{equation}}
\newcommand{\eeq}{\end{equation}}

\newcommand{\ba}{\begin{array}}
\newcommand{\ea}{\end{array}}

\newcommand{\bea}{\begin{eqnarray}}
\newcommand{\eea}{\end{eqnarray}}

\newcommand{\bc}{\begin{center}}
\newcommand{\ec}{\end{center}}

\newcommand{\ds}{\displaystyle}

\newcommand{\bt}{\begin{tabular}}
\newcommand{\et}{\end{tabular}}

\newcommand{\bi}{\begin{itemize}}
\newcommand{\ei}{\end{itemize}}

\newcommand{\bd}{\begin{description}}
\newcommand{\ed}{\end{description}}

\newcommand{\bp}{\begin{pmatrix}}
\newcommand{\ep}{\end{pmatrix}}

\newcommand{\p}{\partial}
\newcommand{\sech}{\mbox{sech}}

\newcommand{\cf}{{\it cf.}~}

\newcommand{\ltwo}{L_{2}(\mathbb{R}^{2})}
\newcommand{\smooth}{C^{\infty}_{0}(\mathbb{R}^{2})}

\newcommand{\br}{{\bf r}}
\newcommand{\bk}{{\bf k}}
\newcommand{\bv}{{\bf v}} 
\newcommand{\bu}{{\bf u}}
\newcommand{\bdd}{{\bf d}}
\newcommand{\bpp}{{\bf p}}
\newcommand{\bm}{{\bf m}}
\newcommand{\bn}{{\bf n}}

\newcommand{\pt}{\mathcal{K}}

\newcommand{\gnorm}[1]{\left|\left| #1\right|\right|}
\newcommand{\Lnorm}{L^{2}\left(\mathbb{R}^{2}\right)}
\newcommand{\ipro}[2]{\left<#1,#2 \right>}
\pagestyle{empty}
\begin{document}

\begin{center}
{\bf Gradescope Assignment: Due 2/24/21\\
0 pts for no work\\ 2 pts for attempt\\ 4 pts for full answer}
\end{center}
By
$$
A{\bf x} = {\bf 0},
$$
we mean the matrix problem
$$
\bp a_{11} & a_{12} \\ a_{21} & a_{22}\ep\bp x_{1}\\x_{2} \ep = \bp 0\\ 0 \ep
$$ 
which can be written in the augmented form
$$
\left(
\ba{cc|c}
a_{11} & a_{12} & 0 \\ 
a_{21} & a_{22} & 0 
\ea
\right)
$$
or component-by component so that 
$$
a_{11}x_{1} + a_{12}x_{2} = 0, ~ a_{21}x_{1} + a_{22}x_{2} = 0.
$$
We define the determinant of $A$, $\text{det}(A)$, to be 
$$
\text{det}(A) = a_{11}a_{22} - a_{12}a_{21}.
$$
\begin{enumerate}
\item (Short) Show that if $A{\bf x}={\bf 0}$ has a nontrivial solution (i.e. $x_{1}$ or $x_{2}$ is not zero), then $\text{det}(A)=0$.  Your solution should address the two cases: 
\begin{enumerate}
\item $x_{1}\neq 0$, so that $a_{11}=-a_{12}x_{2}/x_{1}$ and $a_{21}=-a_{22}x_{2}/x_{1}$ 
\item $x_{2}\neq 0$, so that $a_{12}=-a_{11}x_{1}/x_{2}$ and $a_{22}=-a_{21}x_{1}/x_{2}$ 
\end{enumerate}
\item (Short(ish)) Show that if $\text{det}(A)=0$ then $A{\bf x}=0$ has an infinite number of nontrivial solutions.   You should address the following cases:
\begin{enumerate}
\item $a_{11}\neq 0$, which leads to the row-reduced echelon matrix problem 
$$
\left(
\ba{cc|c}
a_{11} & a_{12} & 0 \\ 
0 & \frac{a_{11}a_{22}-a_{12}a_{21}}{a_{11}} & 0 
\ea
\right)
$$

\item $a_{21}\neq 0$, which leads to the row-reduced matrix problem 
$$
\left(
\ba{cc|c}
0 & \frac{a_{12}a_{21}-a_{11}a_{22}}{a_{21}} & 0 \\ 
a_{21} & a_{22} & 0 
\ea
\right)
$$

\item $a_{21} = 0$ and $a_{11} = 0$, which leads to the augmented matrix problem 
$$
\left(
\ba{cc|c}
0 & a_{12} & 0 \\ 
0 & a_{22} & 0 
\ea
\right)
$$

\end{enumerate}

\end{enumerate}
\end{document}